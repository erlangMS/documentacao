 %% modeloABNT2.tex, v1.0 athila
%% Copyright 2013 by Athila e Monaro
%%
%% Este trabalho Ž uma adequa‹o das normas de disserta›es e teses
%% da Universidade de S‹o Paulo - USP de acordo com a Norma ABNT
%%
%%
%% Quaisquer dœvidas favor enviar e-mail para:
%%
%% athila.uno@gmail.com ou
%% renato.monaro@gmail.comsh 
%%

% ------------------------------------------------------------------------
% ------------------------------------------------------------------------
% eesc: Modelo de Trabalho Acadmico (tese de doutorado, disserta‹o de
% mestrado e trabalhos monogr‡ficos em geral) em conformidade com 
% ABNT NBR 14724:2011. Esta classe estende as funcionalidades da classe
% abnTeX2 elaborada de forma a adequar os par‰metros exigidos pelas 
% normas USP e do departamento de elŽtrica da Escola de Engenharia 
% de S‹o Carlos - USP.
% ------------------------------------------------------------------------
% ------------------------------------------------------------------------

% ------------------------------------------------------------------------
% Op›es:
% 	tesedr:     Formata documento para tese de doutorado
%	qualidr:    Formata documento para qualifica‹o de doutorado
% 	dissertmst: Formata documento para disserta‹o de mestrado
% 	qualimst:   Formata documento para qualifica‹o de mestrado
% ------------------------------------------------------------------------
\documentclass[qualidr]{eesc}

% ---
% PACOTES
% ---

% ---
% Pacotes fundamentais 
% ---
%\RequirePackage[applemac]{inputenc}
\usepackage{cmap}				% Mapear caracteres especiais no PDF
\usepackage{lmodern}				% Usa a fonte Latin Modern			
\usepackage{makeidx}            	% Cria o indice
\usepackage{hyperref}  			% Controla a forma‹o do ’ndice
\usepackage{lastpage}			% Usado pela Ficha catalogr‡fica
\usepackage{indentfirst}			% Indenta o primeiro par‡grafo de cada se‹o.
\usepackage{nomencl} 			% Lista de simbolos
\usepackage{graphicx,url}			% Inclus‹o de gr‡ficos
\usepackage{listings}

% ---

% ---
% Pacotes adicionais, usados apenas no ‰mbito do Modelo eesc
% ---
\usepackage{lipsum}				       % para gera‹o de dummy text
\usepackage[printonlyused]{acronym}
\usepackage[table]{xcolor}
% ---
\usepackage[brazil]{babel} 
\usepackage[utf8]{inputenc}
\usepackage{listingsutf8}


\renewcommand{\citebrackets}[2]{\def\citeopen{#1}\def\citeclose{#2}}
\renewcommand{\setcitebrackets}{\citebrackets[]}

%\usepackage[brazilian,hyperpageref]{backref}	 % Paginas com as citações na bibl
\usepackage[num]{abntex2cite}	% Citações padrão ABNT


%\renewcommand{\backref}{}
% Define os textos da citação
%\renewcommand*{\backrefalt}[4]{
%	\ifcase #1 %
%		Nenhuma citação no texto.%
%	\or
%		Citado na página #2.%
%	\else
%		Citado #1 vezes nas páginas #2.%
%	\fi}%
% ---
% Informa›es de dados para CAPA e FOLHA DE ROSTO
% ---
%
% T’tulo:
%	1. T’tulo em portugus
%	2. T’tulo em ingls
\titulo{Implementação do protocolo SNMP para monitoramento de serviços no Barramento ErlangMS da Universidade de Brasília}{}
%	Linha de Pesquisa: Engenharia de Software}{}
%
% Autor:
%	1. Nome completo do autor
%	2. Formato de nome para bibliografia
\autor{Felipe Evangelista dos Santos\\Fundação Universidade de Brasília}{E. Santos, Felipe}
%
% Cidade
\local{Brasília}
% Ano de defesa
\data{maio, 2017}
% çrea de concentra‹o da pesquisa
\areaconcentracao{Engenharia de Software.}
% Nome do orientador
%\orientador{Prof. Dr. Eduardo Henrique F.M. Teixeira}
% Nome do coorientador
% ---

% ---
% Configurações de aparência do PDF final

% alterando o aspecto da cor azul
\definecolor{blue}{RGB}{41,5,195}

% informações do PDF
\makeatletter
\hypersetup{
     	%pagebackref=true,
		pdftitle={\@title}, 
		pdfauthor={\@author},
    	pdfsubject={\imprimirpreambulo},
	    pdfcreator={LaTeX with abnTeX2},
		pdfkeywords={abnt}{latex}{abntex}{abntex2}{relatório técnico}, 
		colorlinks=true,       		% false: boxed links; true: colored links
    	%liabilitiesnkcolor=blue,          	% color of internal links
    	filecolor=magenta,      		% color of file links
		urlcolor=blue,
		bookmarksdepth=4
}
\makeatother
% --- 
% ---
% compila o indice
% ---
\makeindex
% ---

% ---
% Compila a lista de abreviaturas e siglas
% ---
\makenomenclature
% ---

% ---
% Inserir ficha catalogr‡fica
%
% Caso o comando \inserirfichacatalografica seja definido, a ficha catalogr‡fica
% ser‡ inserida atr‡s da folha de rosto. Caso contr‡rio a p‡gina ser‡ deixada em
% branco.
%
% CUIDADO: Esta op‹o deve ser preenchida antes do comando \maketitle
% ---
%\inserirfichacatalografica{fichaCatalografica.pdf}
% ---

% ---
% Inserir folha de aprova‹o
%
% Caso o comando \inserirfolhaaprovacao seja definido, a a folha de aprova‹o
% ser‡ inserida. AlŽm disso, conforme Resolu‹o CoPGr 5890, as informa›es 
% de rodapŽ s‹o inseridas apropriadamente na folha de rosto.
%
% CUIDADO: Esta op‹o deve ser preenchida antes do comando \maketitle
% ---
%\inserirfolhaaprovacao{folhaAprovacao.pdf}
% ---

% ----
% In’cio do documento
% ----

\begin{document}

% ----------------------------------------------------------
% ELEMENTOS PRƒ-TEXTUAIS
% ----------------------------------------------------------
\pretextual

% ---
% Insere Capa, Folha de rosto, Ficha catalogr‡fica (se inserida)
% e folha de aprova‹o (se inserida).
% ---
\maketitle

% ---
% Dedicat—ria
% ---
%\imprimirdedicatoria{Este trabalho Ž dedicado ˆ minha fam’lia e amigos.}
% ---

% ---
% Agradecimentos
% ---
%\imprimiragradecimentos{
%Os agradecimentos principais s‹o direcionados ˆ minha filha Helo’sa e esposa Vaneide, ao meu irm‹o Ben’cio, minha m‹e Socorro, irm‹ Silvana, Vladmir e a todos os amigos.

%Agradecimentos especiais ˆ equipe de desenvolvimento de software da Sysofit pelo apoio e por fazer parte deste projeto muito ambicioso.
%}
% ---

% ---
% Ep’grafe
% ---
%\imprimirepigrafe{
%		``Somos essencialmente profissionais do sentido. Educamos, \\
%		 quando ensinamos com sentido. Educar Ž impregnar de sentido\\
%		 a vida. A profiss‹o docente est‡ centrada na vida, no bem querer.''\\
%		(Prof. Gilberto Teixeira)
%}
% ---

% ---
% RESUMO e ABSTRACT
% ---

% Resumo em portugus
%\begin{resumo}{\emph{Middleware}, Toler‰ncia a Falhas, Elasticidade, \emph{Cluster}}

  %Este trabalho prop›e um \emph{middleware} para implementar  toler‰ncia a falhas e elasticidade em um \emph{cluster} de alta performance e  alta disponibilidade. Assim sendo, espera-se que o \emph{middleware} proposto seja capaz de  diminuir o nœmero de servidores necess‡rios para processar as filas de mensagens de um sistema distribu’do e, consequentemente, economizar recursos computacionais  para manuten›es evolutivas e corretivas. Para isso Ž necess‡ria uma arquitetura que seja el‡stica o suficiente para se adaptar ao crescimento da fila de requisi›es de forma que as mensagens n‹o se acumulem, e que seja tolerante a falhas para que eventuais paradas do sistema, por queda ou falha dos servios, n‹o impacte na operacionalidade do \emph{cluster}. 


%\end{resumo}

% ---

% ---
% inserir lista de ilustra›es
% ---
%\listailustracoes
% ---

% ---
% inserir lista de tabelas
% ---
%\listatabelas
% ---

% ---
% inserir lista de abreviaturas e siglas
% ---
%\listasiglas{abrev/Abreviaturas}
% ---

% ---
% inserir o sumario
% ---
\sumario
% ---

% ----------------------------------------------------------
% ELEMENTOS TEXTUAIS
% ----------------------------------------------------------
\mainmatter

% ----------------------------------------------------------
% Introdu‹o
% ----------------------------------------------------------
\chapter[Implementação do protocolo SNMP para monitoramento de serviços no Barramento ErlangMS da Universidade de Brasília]{Implementação do protocolo SNMP para monitoramento de serviços no Barramento ErlangMS da Universidade de Brasília}

\section[Introdução]{Introdução}



\section{Justificativa}

Com o aumento da implementação e disponibilização de serviços na UnB, foi identificada a necessidade de um efetivo monitoramento dos serviços, através da coleta de dados ou informações extraídas das requisições. Para gerenciar o monitoramento são necessária ferramentas para um controle mais fácil e objetivo, atualmente o CPD utiliza o Nagios, uma plataforma utilizada para acompanhamento e monitoramento da infraestrutura de redes da UnB,já a parte de sistemas e serviços não são monitoradas de forma precisa ou que possa trazer informações com certa relevância, pois não há uma comunicação ou integração dos sistemas e serviço de forma apropriada, o que implica em um déficit no acompanhamento e monitoramento nos sistemas e serviços. Além disso, também não há um acompanhamento especifico voltado para o monitoramento do ambiente em que as aplicações e serviços estão hospedados. Ou seja, percebe-se que o gerenciamento de importantes funcionalidades são falhos e que precisam ser melhorados.

Dessa forma, com essa pesquisa, espera-se prover meios para realizar a integração do Nagios com o barramento ErlangMS, utilizado o protocolo SNMP para facilitar e tornar o gerenciamento dos serviços mais abrangente contribuindo com um bom funcionamento e acompanhamentos dos \textit{softwares} da UnB. Além disso pretende-se também que a implementação do protocolo SNMP para monitoramento de serviços no barramento ErlangMS possa trazer grandes benefícios como o gerenciamento de falhas, requisições, desempenho e quantidade de acessos em um determinado momento, a partir da implementação, criar e especificar processos, métodos, assim como, realizar estudos e utilizar métricas para estatísticas após a coleta da informações advindas dos serviços implementados para esse propósito. 


\section{Objetivo Geral}

O objetivo geral da pesquisa proposta é implementar protocolo SNMP para monitoramento de serviços no Barramento ErlangMS da UnB. Para isso serão realizadas  pesquisas e projetos desenvolvidos e utilizados para monitoramentos de serviços, técnicas e ferramentas de apoio, visando à melhoria do gerenciamento dos serviços.

\subsection{Objetivos Específicos}

Os objetivos específicos são:

\begin{itemize}
	 
\item Levantamento do estado da arte por meio da análise crítica dos artigos científicos recentes referentes ao tema implementação do protocolo SNMP para monitoramento de serviços no barramento ErlangMS;

\item Definir um processo para a realização da implementação do protocolo SNMP para monitoramento de serviços no barramento ErlangMS baseando-se nas  abordagens mais relevantes estudadas na literatura;

\item Utilizar ferramentas de apoio para a implementação do protocolo SNMP para monitoramento de serviços no barramento ErlangMS da UnB;

\item Realizar um estudo de caso aplicando o método desenvolvido  para promover adequações e melhorias no gerenciamento de monitoramento dos serviços da UnB; 

\item Publicação de artigos científicos contendo descobertas relevantes e os resultados do estudo de caso realizado. 

\end{itemize}

\section{Revisão da Literatura}

Em \cite{Tanenbaum} é apresentada a importância da de definição de um protocolo que é um conjunto de regras que contra o formato e o significado dos pacotes ou mensagens que são trocadas. Em \cite{deGeus} é descrita a definição de um modelo computacional configurável para o gerenciamento e monitoramento de redes, servidores, armazenamento, aplicações e serviços com a utilização do protocolo SNMP para realização de coleta de informações para que sejam criadas métricas onde se possa obter resultados satisfatórios dos serviços disponibilizados. Em sequência, no \cite{daSilva} são explicadas informações do protocolo SNMP, como o seu funcionamento, sua utilização ,Agente(processo), os tipos de Agente, o Gerente que é uma aplicação, em execução em uma estação de gerenciamento, as operações do protocolo, como por exemplo, \textit{GetRequest, 	GetNextRequest,  GetResponse,  SetRequest  e Trap} e também sobre as ferramentas de monitoramento que são compatíveis com o protocolo. 

Em \cite{Dias} é descrita a forma de construção da plataforma de gerenciamento SNMP  utilizando  o RFC - \textit{Request For Comment} e suas definições. Uma abordagem interessante para a implementação dos serviços que utilizarão o protocolo SNMP  descrita em  \cite{Agilar} onde é proposto um modelo de desenvolvimento juntamente, com o Barramento ErlangMS que é responsável pelo serviço de mensageria, peça importante para a comunicação das ferramentas de monitoramento e os serviços implementados. Entretanto, em \cite{deMello} são descritos e identificados alguns pontos fracos do protocolo SNMP. Apesar  de  seu  nome,  \textit{"Simple"  Network  Management  Protocol},  o  SNMP  é  um protocolo  relativamente  complexo  para  implementar.  Também,  o  SNMP  não  é  um protocolo muito eficiente. Os modos nos quais são identificadas as variáveis SNMP (como \textit{strings} de  \textit{byte} onde  cada  \textit{byte} corresponde  a  um  nodo  particular  no  banco  de  dados da MIB) conduz desnecessariamente a grandes pacotes dedados PDU (\textit{Protocol Description Unit}), que consomem partes significativas de cada mensagem de SNMP, sobrecarregando a rede de transmissão de dados.  


Uma outra abordagem é sobre as ferramentas de gerenciamento mais utilizadas no mercado como por exemplo: ZABBIX, NAGIOS e CACTI, todas têm muitas semelhanças entre si, porém algumas são mais funcionais para um tipo de finalidade do que outras. \cite{Braga}. Nesse estudo partes comuns a vários programas ou serviços poderão ser implementados com informações desnecessárias para uma efetivo monitoramento. Entretanto, deve haver uma preocupação com o desempenho e gerenciamento, devido alta construção de serviços genéricos e comuns.


\section{Metodologia}

A implementação do protocolo SNMP para monitoramento de serviços no barramento ErlangMS será realizado no CPD da UnB, e a execução do projeto ocorrerá  na área de desenvolvimento de sistemas do setor Serviço de Sistemas da Informação - SSI. 

Serão coletados dados sobre os serviços executados no barramento ErlangMS, a fim de buscar informações relativas às requisições, solicitações e a execução de agentes e gerentes SNMP. Assim haverá uma análise prévia para a identificação das ferramentas ou aplicações que poderão ser integradas aos serviços implementados para o monitoramento, de forma a encontrar à maneira mais eficiente para atender às situações específicas do gerenciamento.

Faz parte ainda do escopo metodológico do projeto a realização de pesquisas para o entendimento mais aprofundado do protocolo SNMP, seus tipos de requisições , tipos de agentes e as operações do protocolo, ferramentas já utilizadas para o gerenciamento, também sobre o a linguagem Erlang e a montagem de configuração do ambiente de desenvolvimento para a implementação dos serviços. 

Para a melhoria desse entendimento serão utilizados  processos, ferramentas e o \textit{Request for Comments - RFC}, que  é um documento que descreve os padrões de cada protocolo. Um ambiente computacional será disponibilizado para a realização da coleta de dados dos serviços implementados, avaliação do gerenciamento e monitoramento juntamente com a aceitação dos envolvidos no projeto.

\section{Plano de Trabalho e Cronograma}


\begin{table}[!htpb]
	\centering
	\caption{Cronograma de Atividades do Mestrado}
	\begin{center}
		\begin{tabular}{|l|c|c|c|c|}
			\hline
			Tarefa  &2017/2   &2018/1 &2018/2 &2019/1   \\
			\hline\hline
			Disciplinas do Núcleo Básico &X & & & \\
			\hline
			Disciplinas de Engenharia de \textit{Software} & &X & &  \\
			\hline
			Tarefa 1 &X &X &X &  \\
			\hline
			Tarefa 2 & &X &X &  \\
			\hline
			Tarefa 3 & & &X &  \\
			\hline
			Tarefa 4 & &X &X &X  \\
			\hline
			Tarefa 5 & & &X &X  \\
			\hline
			Tarefa 6 & & & &X  \\
			\hline
			Tarefa 7 & &X &X &X  \\
			\hline
			Tarefa 8 & & & &X  \\
			\hline
		\end{tabular}
		\label{tab:cronograma2}
	\end{center}
\end{table} 


% ---





% ---
% Finaliza a parte no bookmark do PDF, para que se inicie o bookmark na raiz
% ---
% ---

% ---
	% Conclus‹o
% \textemdash\ 


% ----------------------------------------------------------
% ELEMENTOS PîS-TEXTUAIS
% ----------------------------------------------------------
%\postextual

% ----------------------------------------------------------
% Referncias bibliogr‡ficas
% ----------------------------------------------------------
%\nocite{*}
\makeatletter
\renewcommand\@biblabel[1]{{\parbox{0.8cm}{[#1]}}}
\makeatother
\bibliography{bib}
% -----------------------
% ----------------------------------------------------------
% Gloss‡rio
% ----------------------------------------------------------
%
%\glossary

\bookmarksetup{startatroot}% 

% ----------------------------------------------------------
% Apndices
% ----------------------------------------------------------
% ---
% Inicia os apndices
% ---

% ---

% ----------------------------------------------------------
% Anexos
% ----------------------------------------------------------
% ---
% Inicia os anexos
% ---

\end{document}

